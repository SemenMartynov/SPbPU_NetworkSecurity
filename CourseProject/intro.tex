\chapter*{Введение}
\addcontentsline{toc}{chapter}{Введение}

Технология WebSocket представляет собой протокол двунаправленной связи между клиентом и сервером, который позволяет устанавливать постоянное соединение и обмениваться данными в реальном времени. В отличие от традиционного подхода, где клиент отправляет запрос на сервер, а сервер отвечает, WebSockets позволяют обоим сторонам инициировать передачу данных без задержек и лишних запросов.

Если сравнивать WebSocket с традиционными сокетами (Berkeley Socket), можно отметить что эта технология имеет более короткую историю. Формальный стандарт, описывающий протокол WebSocket и его реализацию, был опубликован рабочей группой IETF (Internet Engineering Task Force) в документе "The WebSocket Protocol" (RFC 6455) только в июле 2011 года.

WebSockets основан на протоколе HTTP и использует специальный заголовок для установки соединения между клиентом и сервером. После установки соединения, клиент и сервер могут отправлять сообщения друг другу в виде бинарных или текстовых данных. Это делает WebSockets идеальным выбором для различных приложений, где требуется обновление информации в реальном времени, таких как чаты, онлайн-игры, мониторинг и др.

Основные преимущества WebSockets включают:
\begin{itemize}
\item Низкая задержка: WebSockets позволяют устанавливать постоянное соединение, что устраняет необходимость в повторных запросах и сокращает задержку передачи данных.
\item Эффективное использование ресурсов: Постоянное соединение WebSockets требует меньше ресурсов сервера и сети по сравнению с частыми запросами и ответами HTTP.
\item Двунаправленная связь: WebSockets позволяют обмениваться данными между клиентом и сервером в обоих направлениях, что открывает возможности для интерактивного взаимодействия и обновления информации.
\item Расширяемость: WebSockets поддерживают расширение протокола, что позволяет добавлять собственные функции и возможности поверх стандартного протокола.
\end{itemize}

На сегодня, WebSockets остаются популярной технологией веб-разработки и широко применяются для создания интерактивных и реального времени веб-приложений. В данной работе мы проведём исследование основных возможностей этого протокола, рассмотрим его устройство и приведём практические примеры использования.

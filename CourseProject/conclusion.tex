\chapter*{Заключение}
\addcontentsline{toc}{chapter}{Заключение}

В ходе проведенного исследования были рассмотрены основные возможности протокола WebSocket, его устройство и приведены практические примеры использования. Целью работы было изучить протокол WebSocket и понять его преимущества и особенности по сравнению с другими технологиями для обмена данными в режиме реального времени.

Основные принципы WebSocket, такие как двунаправленная связь, инициация соединения сервером, поддержка реального времени, эффективное использование ресурсов и совместимость с прокси-серверами и брандмауэрами, делают его идеальным выбором для приложений, требующих непрерывного обмена данными между клиентом и сервером.

Архитектура WebSocket, включающая установку и завершение соединения, фреймы и их структуру, а также WebSocket API, предоставляет гибкую платформу для разработки приложений реального времени. Протокол WebSocket также был сравнен с другими технологиями, такими как HTTP-поллинг, Server-Sent Events (SSE) и WebRTC, и показал высокий уровень в области производительности и эффективности.

В работе были представлены практические примеры использования WebSocket, включая реализацию простого чата на платформе node.js и создание тестера скорости соединения на языке программирования Rust. Эти примеры демонстрируют, как легко и эффективно можно использовать протокол WebSocket для различных типов приложений.

Выводы по работе:
\begin{enumerate}
\item Протокол WebSocket - полезная и востребованная технология для реализации обмена данными между клиентом и сервером в режиме реального времени. Это достигается благодаря двунаправленной связи, низкому времени задержки и эффективному использованию ресурсов.
\item WebSocket обладает рядом преимуществ перед другими технологиями, что делает его актуальным выбором для решения задач, связанных с обменом данными в режиме реального времени.
\item Несмотря на свои преимущества, важно учесть возможные риски, связанные с безопасностью и производительностью при использовании WebSocket в конкретном проекте, и применить соответствующие меры для смягчения этих рисков.
\item Практические примеры использования WebSocket успешно применяются для решения различных задач, свидетельствующих об универсальности и вариативности этой технологии.
\end{enumerate}

Таким образом, в результате исследования и анализа данного протокола можно сделать вывод о его значимости для обеспечения непрерывного и эффективного обмена данными в режиме реального времени. Его преимущества, такие как низкая задержка, масштабируемость, поддержка расширений и простота использования, делают его идеальным выбором для разработки современных веб-приложений. Протокол WebSocket открывает новые возможности в области интерактивных веб-приложений, многопользовательских игр, финансовых торговых платформ и других приложений, где важна своевременная доставка данных и низкая задержка взаимодействия.
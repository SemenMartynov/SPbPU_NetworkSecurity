\chapter*{Лабораторная работа 4. Импорт и экспорт ключей. Цифровая подпись}
\addcontentsline{toc}{chapter}{Лабораторная работа 4. Импорт и экспорт ключей. Цифровая подпись}

\textbf{Цель работы:} Знакомство с возможностями утилиты GNU Privacy Guard.

\section*{1. Экспорт открытого ключа}
\addcontentsline{toc}{section}{1. Экспорт открытого ключа}

Экспорт открытого ключа в текстовый файл
\begin{Verbatim}[frame=single,breaklines=true,breakanywhere=true]
    smart@thinkpad$ gpg2 \
     --armor \
     --export 9DAF94FCD9CA38BFD298BD0CA0B01E0BAB2AF7C6 > mykey.asc
\end{Verbatim}

\lstinputlisting[caption={Открытый ключ mykey.asc}, numbers=none]
{../Tasks/3_4_GnuPG_tool_utilisation/source_files/mykey.asc}

\section*{2. Создание электронной цифровой подписи (ЭЦП) файла}
\addcontentsline{toc}{section}{2. Создание электронной цифровой подписи (ЭЦП) файла}

Отсоединённая ЭЦП файла \texttt{mydocument.pdf} в текстовом формате (команда требует ввода ключевой фразы)
\begin{Verbatim}[frame=single,breaklines=true,breakanywhere=true]
    smart@thinkpad$ gpg2 \
     --sign \
     --detach-sign \
     --default-key 9DAF94FCD9CA38BFD298BD0CA0B01E0BAB2AF7C6 \
     --armor mydocument.pdf
\end{Verbatim}

\lstinputlisting[caption={Отсоединёння цифровая подпись для файла mydocument.pdf}, numbers=none]
{../Tasks/3_4_GnuPG_tool_utilisation/source_files/DigitalSignature/mydocument.pdf.asc}

Отсоединённая подпись в двоичном формате

\begin{Verbatim}[frame=single,breaklines=true,breakanywhere=true]
    smart@thinkpad$ gpg2 \
     --sign \
     --detach-sign \
     --default-key 9DAF94FCD9CA38BFD298BD0CA0B01E0BAB2AF7C6 \
     mydocument.pdf
\end{Verbatim}

Листинг двоичного файла не имеет смысла.

Встроенная в файл подпись в текстовом формате

\begin{Verbatim}[frame=single,breaklines=true,breakanywhere=true]
    smart@thinkpad$ gpg2 \
     --sign \
     --default-key 9DAF94FCD9CA38BFD298BD0CA0B01E0BAB2AF7C6 \
     --armor mydocument.pdf
\end{Verbatim}

Листинг не имеет смысла, т.к. включает объёмный исходный PDF-файл.

Встроенная в файл подпись в двоичном формате

\begin{Verbatim}[frame=single,breaklines=true,breakanywhere=true]
    smart@thinkpad$ gpg2 \
     --sign \
     --default-key 9DAF94FCD9CA38BFD298BD0CA0B01E0BAB2AF7C6 \
     mydocument.pdf
\end{Verbatim}

Листинг не имеет смысла, т.к. сгенерирован бинарный файл.

\section*{3. Импорт открытого ключа}
\addcontentsline{toc}{section}{3. Импорт открытого ключа}

Возьмём файл цифровой подписи GnuPG проекта файлового шифровальщика VeraCrypt.

\begin{Verbatim}[frame=single,breaklines=true,breakanywhere=true]
    smart@thinkpad$ wget https://www.idrix.fr/VeraCrypt/VeraCrypt_PGP_public_key.asc
\end{Verbatim}

Перед импортом, можно удостовериться, что цифровая подпись содержит правильный ID.
\begin{Verbatim}[frame=single,breaklines=true,breakanywhere=true]
    smart@thinkpad$ gpg --show-keys VeraCrypt_PGP_public_key.asc
    pub   rsa4096 2018-09-11 [SC]
        5069A233D55A0EEB174A5FC3821ACD02680D16DE
    uid                      VeraCrypt Team (2018 - Supersedes Key ID=0x54DDD393) <veracrypt@idrix.fr>
    sub   rsa4096 2018-09-11 [E]
    sub   rsa4096 2018-09-11 [A]
\end{Verbatim}

После этого можно импортировать файл
\begin{Verbatim}[frame=single,breaklines=true,breakanywhere=true]
    smart@thinkpad$ gpg --import VeraCrypt_PGP_public_key.asc 
    gpg: key 821ACD02680D16DE: 1 signature not checked due to a missing key
    gpg: key 821ACD02680D16DE: public key "VeraCrypt Team (2018 - Supersedes Key ID=0x54DDD393) <veracrypt@idrix.fr>" imported
    gpg: Total number processed: 1
    gpg:               imported: 1
    gpg: marginals needed: 3  completes needed: 1  trust model: pgp
    gpg: depth: 0  valid:   1  signed:   0  trust: 0-, 0q, 0n, 0m, 0f, 1u
    gpg: next trustdb check due at 2023-03-04
\end{Verbatim}

Для проверки подписи \underline{не обязательно} устанавливать доверительные отношения с импортированным ключом.

\section*{4. Проверка ЭЦП}
\addcontentsline{toc}{section}{4. Проверка ЭЦП}

Для проверки, возьмём архив с исходными кодами и отсоединённую подпись в двоичном формате
\begin{Verbatim}[frame=single,breaklines=true,breakanywhere=true]
    smart@thinkpad$ wget https://launchpad.net/veracrypt/trunk/1.25.9/+download/VeraCrypt_1.25.9_Source.tar.bz2
    smart@thinkpad$ wget https://launchpad.net/veracrypt/trunk/1.25.9/+download/VeraCrypt_1.25.9_Source.tar.bz2.sig
\end{Verbatim}

Теперь всё готово для проверки.
\begin{Verbatim}[frame=single,breaklines=true,breakanywhere=true]
    smart@thinkpad$ gpg --verify VeraCrypt_1.25.9_Source.tar.bz2.sig VeraCrypt_1.25.9_Source.tar.bz2
    gpg: Signature made Sun 20 Feb 2022 04:18:32 PM MSK
    gpg:                using RSA key 5069A233D55A0EEB174A5FC3821ACD02680D16DE
    gpg: Good signature from "VeraCrypt Team (2018 - Supersedes Key ID=0x54DDD393) <veracrypt@idrix.fr>" [unknown]
    gpg: WARNING: This key is not certified with a trusted signature!
    gpg:          There is no indication that the signature belongs to the owner.
    Primary key fingerprint: 5069 A233 D55A 0EEB 174A  5FC3 821A CD02 680D 16DE
\end{Verbatim}

Ожидаемый результат -- \texttt{Good signature}.

Остальные предупреждения, связанные с отсутствием доверия, можно игнорировать.

\section*{5. Экспорт/импорт на ключевые сервера}
\addcontentsline{toc}{section}{5. Экспорт/импорт на ключевые сервера}

Экспорт своего ключа
\begin{Verbatim}[frame=single,breaklines=true,breakanywhere=true]
    smart@thinkpad$ gpg --keyserver hkp://keyserver.ubuntu.com --send-key 9DAF94FCD9CA38BFD298BD0CA0B01E0BAB2AF7C6
    gpg: sending key A0B01E0BAB2AF7C6 to hkp://keyserver.ubuntu.com
\end{Verbatim}

Теперь воспользуемся поиском, для обнаружения этого ключа на удалённом сервере
\begin{Verbatim}[frame=single,breaklines=true,breakanywhere=true]
    smart@thinkpad$ gpg --keyserver hkp://keyserver.ubuntu.com --search-keys martynov.sa@edu.spbstu.ru
    gpg: data source: http://162.213.33.9:11371
    (1)	Semen Martynov (Test digital sirnature) <martynov.sa@edu.spbstu.ru>
    	  4096 bit RSA key A0B01E0BAB2AF7C6, created: 2023-02-11
    Keys 1-1 of 1 for "martynov.sa@edu.spbstu.ru".  Enter number(s), N)ext, or Q)uit >

    smart@thinkpad$ gpg --keyserver hkp://keyserver.ubuntu.com --list-sigs martynov.sa@edu.spbstu.ru
    pub   rsa4096 2023-02-11 [SC] [expires: 2023-03-04]
          9DAF94FCD9CA38BFD298BD0CA0B01E0BAB2AF7C6
    uid           [ultimate] Semen Martynov (Test digital sirnature) <martynov.sa@edu.spbstu.ru>
    sig 3        A0B01E0BAB2AF7C6 2023-02-11  Semen Martynov (Test digital sirnature) <martynov.sa@edu.spbstu.ru>
    sub   rsa4096 2023-02-11 [E] [expires: 2023-03-04]
    sig          A0B01E0BAB2AF7C6 2023-02-11  Semen Martynov (Test digital sirnature) <martynov.sa@edu.spbstu.ru>
\end{Verbatim}

Так же легко можно импортировать ключи с удалённого сервера
\begin{Verbatim}[frame=single,breaklines=true,breakanywhere=true]
    smart@thinkpad$ gpg --keyserver pgp.mit.edu --recv-keys DAD95197
    gpg: key FBF1FC87DAD95197: public key "Ashish Aniyan <aaniyan@verimatrix.com>" imported
    gpg: Total number processed: 1
    gpg:               imported: 1
\end{Verbatim}

Для интеграции GPG заработал в почтовом клиенте \texttt{Mutt}, нужно в конфигурационный файл \texttt{~/.muttrc дописать}
\begin{Verbatim}[frame=single,breaklines=true,breakanywhere=true]
    set crypt_use_gpgme=yes
    set crypt_autosign=yes
    set crypt_replyencrypt=yes
\end{Verbatim}

\section*{Выводы}
\addcontentsline{toc}{section}{Выводы}

В данной работе мы сравнили различные виды электронных цифровых подписей:
\begin{itemize}
    \item Отсоединённую ЭЦП в текстовом формате
    \item Отсоединённую ЭЦП в бинарном формате
    \item Присоединенную ЭЦП в текстовом формате
    \item Присоединенную ЭЦП в бинарном формате
\end{itemize}

Работа с отсоединённой подписью удобнее. В цифровом виде подпись занимает меньше места.

В практическом плане, использование GNU Privacy Guard позволяет верифицировать не только документы, но и цифровые пакеты при распространении через Интернет.

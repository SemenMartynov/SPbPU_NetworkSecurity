\chapter*{Лабораторная работа 8. Сетевое экранирование. Применение правил iptables}
\addcontentsline{toc}{chapter}{Лабораторная работа 8. Сетевое экранирование. Применение правил iptables}

\textbf{Цель работы:} 

\section*{1. Определение IP-адреса}
\addcontentsline{toc}{section}{1. Определение IP-адреса}

Выполним подключение по ssh к удалённому хосту и запросим его IP адрес с помощью утилиты \texttt{ip}.

\begin{Verbatim}[frame=single]
    user@ubuntu:~$ ip a
    1: lo: <LOOPBACK,UP,LOWER_UP> mtu 65536 qdisc noqueue state UNKNOWN group default qlen 1000
        link/loopback 00:00:00:00:00:00 brd 00:00:00:00:00:00
        inet 127.0.0.1/8 scope host lo
        valid_lft forever preferred_lft forever
        inet6 ::1/128 scope host 
        valid_lft forever preferred_lft forever
    2: enp0s3: <BROADCAST,MULTICAST,UP,LOWER_UP> mtu 1500 qdisc pfifo_fast state UP group default qlen 1000
        link/ether 08:00:27:2a:f7:85 brd ff:ff:ff:ff:ff:ff
        inet 192.168.124.45/24 metric 100 brd 192.168.124.255 scope global dynamic enp0s3
        valid_lft 24307sec preferred_lft 24307sec
        inet6 fe80::a00:27ff:fe2a:f785/64 scope link 
        valid_lft forever preferred_lft forever
\end{Verbatim}

Интересующий нас адрес -- \texttt{192.168.124.45}.

\section*{2. Просмотр текущие правил}
\addcontentsline{toc}{section}{2. Просмотр текущие правил}

Список текущих правил можно получить командой\\\texttt{iptables --list --numeric --verbose --line-numbers} (требует прав суперпользователя), где:

\begin{itemize}
    \item \texttt{--list} -- показать все правила в выбранной цепочке (если цепочка не выбрана, то во всех цепочках);
    \item \texttt{--numeric} -- выводить IP-адреса и номера портов в числовом виде предотвращая попытки преобразовать их в символические имена;
    \item \texttt{--verbose} -- увеличить подробность сообщений (имена интерфейсов, параметры правил, маски TOS, счётчики);
    \item \texttt{--line-numbers} -- вывода номеров строк соответствующих позиции правила в цепочке.
\end{itemize}

\begin{Verbatim}[frame=single]
    user@ubuntu:~$ sudo iptables --list --numeric --verbose --line-numbers
    Chain INPUT (policy ACCEPT 0 packets, 0 bytes)
    num   pkts bytes target     prot opt in     out   source    destination

    Chain FORWARD (policy ACCEPT 0 packets, 0 bytes)
    num   pkts bytes target     prot opt in     out   source    destination

    Chain OUTPUT (policy ACCEPT 0 packets, 0 bytes)
    num   pkts bytes target     prot opt in     out   source    destination
\end{Verbatim}

На данном этапе все цепочки пусты.

\section*{3. Блокировка входящего трафика}
\addcontentsline{toc}{section}{3. Блокировка входящего трафика}

Заблокируем весь входящий трафик, модифицировав таблицу \texttt{INPUT}

\begin{Verbatim}[frame=single]
    user@ubuntu:~$ sudo iptables --append INPUT --jump DROP
\end{Verbatim}

Это естественным образом привело к "зависанию" терминала и обрыву сессии по таймауту.

Новое соединение так же установить нелья, оно отваливается по таймауту.

\begin{Verbatim}[frame=single]
    smart@thinkpad$ ssh user@192.168.124.45 -vvvv
    OpenSSH_9.2p1, OpenSSL 3.0.8 7 Feb 2023
    debug1: Reading configuration data /home/smart/.ssh/config
    debug1: /home/smart/.ssh/config line 9: Applying options for *
    debug1: Reading configuration data /etc/ssh/ssh_config
    debug2: resolve_canonicalize: hostname 192.168.124.45 is address
    debug3: expanded UserKnownHostsFile '~/.ssh/known_hosts' -> '/home/smart/.ssh/known_hosts'
    debug3: expanded UserKnownHostsFile '~/.ssh/known_hosts2' -> '/home/smart/.ssh/known_hosts2'
    debug1: auto-mux: Trying existing master
    debug1: Control socket "/tmp/ssh-master-socket-user@192.168.124.45:22" does not exist
    debug3: ssh_connect_direct: entering
    debug1: Connecting to 192.168.124.45 [192.168.124.45] port 22.
    debug3: set_sock_tos: set socket 3 IP_TOS 0x48
    debug1: connect to address 192.168.124.45 port 22: Connection timed out
    ssh: connect to host 192.168.124.45 port 22: Connection timed out
\end{Verbatim}

Запросы ping до удалённого хоста не проходят

\begin{Verbatim}[frame=single]
    smart@thinkpad$ ping 192.168.124.45
    PING 192.168.124.45 (192.168.124.45) 56(84) bytes of data.
    ^C
    --- 192.168.124.45 ping statistics ---
    283 packets transmitted, 0 received, 100% packet loss, time 285758ms
\end{Verbatim}

Сканер Nmap не может обнаружить открыте порты (но видит MAC-адрес сетевого интерфейса)

\begin{Verbatim}[frame=single]
    smart@thinkpad$ sudo nmap -sO 192.168.124.45
    Starting Nmap 7.93 ( https://nmap.org ) at 2023-02-16 12:29 MSK
    Nmap scan report for 192.168.124.45
    Host is up (0.00020s latency).
    All 256 scanned ports on 192.168.124.45 are in ignored states.
    Not shown: 256 open|filtered n/a protocols (no-response)
    MAC Address: 08:00:27:2A:F7:85 (Oracle VirtualBox virtual NIC)

    Nmap done: 1 IP address (1 host up) scanned in 6.40 seconds
\end{Verbatim}

С удалённого хоста тоже не получается обратиться ко внешней сети
\begin{Verbatim}[frame=single]
    user@ubuntu$ ping ya.ru
    ^C
    ~
    user@ubuntu$ ping 8.8.8.8
    PING 8.8.8.8 (8.8.8.8) 56(84) bytes of data.
    ^C
    --- 8.8.8.8 ping statistics ---
    5 packets transmitted, 0 received, 100% packet loss, time 4080ms
\end{Verbatim}

Теперь посмотрим правила iptables на удалённой машине, и отметим увеличение счётчика заблокированных пакетов.
\begin{Verbatim}[frame=single]
    user@ubuntu$ sudo iptables --list --numeric --verbose --line-numbers
    Chain INPUT (policy ACCEPT 0 packets, 0 bytes)
    num   pkts bytes target     prot opt in   out   source      destination
    1      267 65107 DROP       all  --  *    *     0.0.0.0/0   0.0.0.0/0

    Chain FORWARD (policy ACCEPT 0 packets, 0 bytes)
    num   pkts bytes target     prot opt in   out   source      destination

    Chain OUTPUT (policy ACCEPT 0 packets, 0 bytes)
    num   pkts bytes target     prot opt in   out   source      destination
\end{Verbatim}

\section*{4. Фильтрация входящего трафика}
\addcontentsline{toc}{section}{4. Фильтрация входящего трафика}

Тут мы предполагаем, что на удалённом сервер работают следующие сервисы:
\begin{itemize}
    \item DNS: 53/udp -- стандартный порт
    \item DNS: 53/tcp -- RFC описывает случаи, когда DNS может/должен переходить на TCP
    \item WEB: 80/tcp -- стандартный порт http
    \item WEB: 443/tcp -- стандартный порт https
    \item WEB: 443/udp -- стандартный порт https (http3)
\end{itemize}

Установим \texttt{DROP} в качестве политики по умолчанию:\\
\texttt{sudo iptables --policy INPUT DROP}

Разрешим свободное хождение трафика на локальном интерфейсе:\\
\texttt{sudo iptables --append INPUT --in-interface lo --jump ACCEPT}

Разрешим DNS подключения:\\
\texttt{sudo iptables --append INPUT --protocol udp --dport 53 --jump ACCEPT}\\
\texttt{sudo iptables --append INPUT --protocol tcp --dport 53 --jump ACCEPT}

Ответы, от вышестоящих DNS-серверов:\\
\texttt{sudo iptables --append INPUT --protocol udp --sport 53 --dport 1024:65535 --jump ACCEPT}\\
\texttt{sudo iptables --append INPUT --protocol tcp --sport 53 --dport 1024:65535 --jump ACCEPT}

Запросы для Web-сервера:\\
\texttt{sudo iptables --append INPUT --protocol tcp --match multiport --dports 80,443 --jump ACCEPT}\\
\texttt{sudo iptables --append INPUT --protocol udp --dport 443 --jump ACCEPT}

Итоговые правила выглядят следующим образом
\begin{Verbatim}[frame=single]
    user@ubuntu$ sudo iptables --list --numeric --verbose --line-numbers
    Chain INPUT (policy DROP 0 packets, 0 bytes)
    num pkts bytes target  prot opt in  out  source     destination
    1      0     0 ACCEPT  all  --  lo  *    0.0.0.0/0  0.0.0.0/0  
    2      0     0 ACCEPT  udp  --  *   *    0.0.0.0/0  0.0.0.0/0   udp dpt:53
    3      0     0 ACCEPT  tcp  --  *   *    0.0.0.0/0  0.0.0.0/0   tcp dpt:53
    4      0     0 ACCEPT  udp  --  *   *    0.0.0.0/0  0.0.0.0/0   udp spt:53 dpts:1024:65535
    4     17  3170 ACCEPT  udp  --  *   *    0.0.0.0/0  0.0.0.0/0   udp spt:53 dpts:1024:65535
    5      0     0 ACCEPT  tcp  --  *   *    0.0.0.0/0  0.0.0.0/0   tcp spt:53 dpts:1024:65535
    6      7   366 ACCEPT  tcp  --  *   *    0.0.0.0/0  0.0.0.0/0   multiport dports 80,443
    7      0     0 ACCEPT  udp  --  *   *    0.0.0.0/0  0.0.0.0/0   udp dpt:443

    Chain FORWARD (policy ACCEPT 0 packets, 0 bytes)
    num pkts bytes target  prot opt in  out  source     destination

    Chain OUTPUT (policy ACCEPT 0 packets, 0 bytes)
    num pkts bytes target  prot opt in  out  source     destination
\end{Verbatim}

Для эмуляции работы WEB-сервера, используем Netcat.

На стороне сервера:
\begin{Verbatim}[frame=single]
    user@ubuntu$ sudo nc -l 80
    1
    2
    3
\end{Verbatim}

На стороне клиента:
\begin{Verbatim}[frame=single]
    smart@thinkpad$ nc 192.168.124.45 80
    1
    2
    3
\end{Verbatim}

Передача происходит без потерь.

\section*{Выводы}
\addcontentsline{toc}{section}{Выводы}

В этой работе мы познакомились с основными возможностями по управлению цепочками в iptables.

По итогам работы, мы получили сервер, защищённый от внешних подключений.


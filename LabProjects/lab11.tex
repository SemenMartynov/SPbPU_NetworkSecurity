\chapter*{Лабораторная работа 11. Сниффер заголовков сообщений протоколов уровней L2 и L3 модели OSI}
\addcontentsline{toc}{chapter}{Лабораторная работа 11. Сниффер заголовков сообщений протоколов уровней L2 и L3 модели OSI}

\textbf{Цель работы:} Исследуйте возможности RAW-сокетов предоставляющих доступ к полям заголовков сообщений протоколов уровней L2 и L3 модели OSI.

\section*{Введение}
\addcontentsline{toc}{section}{Введение}

Исследование возможностей RAW-сокетов, предоставляющих доступ к полям заголовков сообщений протоколов уровней L2 и L3 модели OSI, является одной из ключевых тем в области сетевых технологий. RAW-сокеты представляют собой механизм, который позволяет разработчикам программного обеспечения взаимодействовать непосредственно с сетевым стеком операционной системы, минуя обычные сетевые протоколы высокого уровня.

В рамках модели OSI, уровень L2 (Data Link Layer) отвечает за передачу данных между соседними узлами сети, обеспечивая физическую адресацию (MAC-адреса) и управление доступом к среде передачи данных. На этом уровне работают протоколы Ethernet, Wi-Fi и другие. RAW-сокеты позволяют программам обрабатывать пакеты на уровне L2, получая доступ к полям заголовка Ethernet-кадра, например, исследовать и управлять MAC-адресами или управлять параметрами фреймов.

Уровень L3 (Network Layer) отвечает за маршрутизацию и доставку данных между сетями. Здесь работают протоколы IP, ICMP, IPv6 и другие. RAW-сокеты, предоставляющие доступ к L3-заголовкам, позволяют программам анализировать и модифицировать IP-адреса, проверять контрольные суммы и другие параметры протокола IP, а также работать с другими протоколами на уровне L3.

Использование RAW-сокетов может быть полезным во множестве сценариев, таких как разработка сетевых приложений, отладка и тестирование сетевых протоколов, реализация сетевых утилит и многого другого. Однако следует отметить, что работа с RAW-сокетами требует определенных привилегий и может быть ограничена на некоторых платформах или в некоторых сетевых окружениях с целью обеспечения безопасности и предотвращения злоумышленнической деятельности.

\section*{1. Разработка приложения}
\addcontentsline{toc}{section}{1. Разработка приложения}

Требования к приложению состоят в разработке и отладке консольное приложение, обладающего возможностями сниффера пакетов, используя технологию RAW-сокетов. Необходимо ориентироваться на сообщения протоколов ARP, ICMP, UDP, TCP.Для разработки был выбран язык разработки Rust с целью исследования его возможностей в данной области.

\lstinputlisting[language=C++, caption={Исходный код сниффера}]
{../simplesniffer/src/main.rs}

\section*{2. Демонстрация работы}
\addcontentsline{toc}{section}{2. Демонстрация работы}

Демонстрация возможности перехвата сообщений отдельных протоколов с помощью разработанного приложения представлена в следующем логе. Она показывает не все возможности приложения (захватываются только 5 случайных пакетов).

Отдельно стоит отметить, что доступ к RAW сокетам требует уровень доступа супер-пользователя, таким образом, сниффер должен запускаться от root.

\lstinputlisting[language=C++, caption={Лог работы сниффера: захват 5 случайных пакетов}]
{../simplesniffer/log.txt}

Мы отказались от фильтрации протоколов и пакетов через параметры запускаемого сниффера и разбираем все пакеты, которые удаётся обнаружить на сетевом интерфейсе.

\section*{Выводы}
\addcontentsline{toc}{section}{Выводы}

В рамках этой работы мы ставили задачу исследования возможностей предоставления доступа к полям заголовков сообщений через RAW-сокеты.

Если сопоставьте наши результаты с результатами работы других общеизвестных снифферов, можно отметить что разница в значительной части связана с интерфейсом, т.к. нижележащие библиотеки и системные вызовы используются одни и те же.

Касательно языка Rust можно отметить что его механика pattern matching крайне удобна для подобных задач. По мере накопления опыта в этом языке, можно перейти от процедурного подхода в коде сниффера, к чему-то более поддерживаемому.

В целом, исследование возможностей RAW-сокетов на уровнях L2 и L3 модели OSI представляет собой интересную и важную область, которая позволяет разработчикам более гибко и эффективно управлять сетевыми операциями на низком уровне и создавать инновационные решения в сетевых технологиях.

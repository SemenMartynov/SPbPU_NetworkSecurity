\chapter*{Лабораторная работа 2. Сокеты L4}
\addcontentsline{toc}{chapter}{Лабораторная работа 2. Сокеты L4}

\textbf{Цель работы:} Создание клиент-серверных приложений, взаимодействующих друг с другом по сети на основе технологии соединения на сокетах L4.

\section*{1. Анализ кода}
\addcontentsline{toc}{section}{1. Анализ кода}
\textbf{Задача:} Проанализируйте код программы \texttt{server\_game.cpp}, иллюстрирующей обмен данными с клиентскими приложениями по итеративной схеме.

\textbf{Ход решения:} Используется сокет семейства \texttt{AF\_INET}. Сервер ожидает соединения на любом IP адресе (\texttt{INADDR\_ANY}), но нам достаточно для подключения \texttt{localhost}. Для соединения будет использоваться порт 1066. Игра работает в один поток.

\section*{2. Компиляция и запуск}
\addcontentsline{toc}{section}{2. Компиляция и запуск}
\textbf{Задача:} Скомпилируйте и запустите \texttt{server\_game}.

\textbf{Эксперимент:} запуск сервера.
\begin{Verbatim}[frame=single]
    smart@thinkpad$ g++ server_game.cpp -o server_game
    smart@thinkpad$ ./server_game 
    start to listen

\end{Verbatim}

Системный журнал зафиксировал выбор слова сервером.
\begin{Verbatim}[frame=single]
    smart@thinkpad$ journalctl -a | grep server_game
    Feb 08 23:43:25 thinkpad server_game[103180]: server_game chose word green
    Feb 08 23:43:40 thinkpad server_game[103180]: server_game chose word green
\end{Verbatim}

\section*{3. Анализ состояния сокета}
\addcontentsline{toc}{section}{3. Анализ состояния сокета}
\textbf{Задача:} Запустите другой терминал и проверьте с него наличие в системе созданного сервером сокета и то, что он находится в состоянии \texttt{LISTEN}. Для этого выполните команду \texttt{netstat -a | grep 1066}.

Проанализируйте вывод данной команды и объясните ее смысл.

\textbf{Ход решения:} Так как команда \texttt{netstat} устарела и была заменена на \texttt{ss}, то будет рассматривать вывод \texttt{ss -atp | grep game}

\textbf{Эксперимент:} результат запуска команды \texttt{ss}
\begin{Verbatim}[frame=single]
    smart@thinkpad$ ss -atp | grep game
    State  Recv-Q Send-Q Local Address:Port       Peer Address:PortProcess
    LISTEN 0      5            0.0.0.0:fpo-fns         0.0.0.0:*    users:(("server_game",pid=103180,fd=3))
\end{Verbatim}

В результате мы видим, что в данный момент сокет находится в состоянии \texttt{LISTEN}. В очереди на получение находятся 0 покетов, на отправку 5. Сервер слушает на любом адресе, но вместо номера порта мы видим \texttt{fpo-fns}. Порт 1066 зарезервирован в системе для какой-то компьютерной игры, поэтому мы видем название вместо номера. Дальше мы видим, что клиент пока не подключился. А в самом конце строки общую информацию о процессе.

\section*{4. Присоединенные сокеты}
\addcontentsline{toc}{section}{4. Присоединенные сокеты}
\textbf{Задача:} Запустите в качестве клиентского процесса утилиту \texttt{telnet} с параметрами: \texttt{telnet localhost 1066}. При организации коммуникации по сети на разных компьютерах вместо \texttt{localhost} при запуске клиента указывается IP-адрес компьютера, на котором был запущен сервер.

\textbf{Ход решения:} Ввиду устаревания \texttt{telnet}, воспользуемся утилитой \texttt{gnu netcat}.

\textbf{Эксперимент:} Запуск клиента
\begin{Verbatim}[frame=single]
    smart@thinkpad$ nc localhost 1066
    :laying on host: h
\end{Verbatim}
Программа запустилось, соединение установленго. Но имя хости на сервере ничем не инициализированно.

\section*{5. Игра}
\addcontentsline{toc}{section}{5. Игра}
\textbf{Задача:} Диалог с сервером заключается в угадывании слова. Оно вводится по буквам с клиентского терминала. При этом сервер вместо неугаданных букв выдает символы \texttt{”-”}, а также считает число оставшихся неудачных попыток (всего их предусмотрено 12).

\textbf{Эксперимент:} Запуск клиента
\begin{Verbatim}[frame=single]
    smart@thinkpad$ nc localhost 1066
    :laying on host: h
    
     -----   12
    ф
     -----   11
    g
     g----   11
    r
     gr---   11
    e
     gree-   11
    e
     gree-   11
    b
     gree-   10
    n
     green   10
    
     green   9
\end{Verbatim}

Заметно, что логика игры не доделана: победная ситуация не обрабатывается корректно.

\section*{6. Состояние сокета в разные моменты}
\addcontentsline{toc}{section}{6. Состояние сокета в разные моменты}
\textbf{Задача:} Завершите серверное приложение с помощью сигнала \texttt{kill}, и затем определите командой \texttt{netstat -a | grep 1066}, когда исчезает из системы соединение на сокетах. Во время сеанса обмена также примените команду \texttt{netstat -a | grep 1066}, чтобы исследовать состояние соединения.

\textbf{Эксперимент:} Состояние сокета после завершения игры
\begin{Verbatim}[frame=single]
    smart@thinkpad$ ss -atp | grep game
    State  Recv-Q Send-Q Local Address:Port       Peer Address:PortProcess
\end{Verbatim}
Сокет не обнаружен.

Состояние сокета в процессе игры
\begin{Verbatim}[frame=single]
    smart@thinkpad$ ss -atp | grep game
    State  Recv-Q Send-Q Local Address:Port       Peer Address:PortProcess
    LISTEN 0      5            0.0.0.0:fpo-fns         0.0.0.0:*         users:(("server_game",pid=109301,fd=3))
    ESTAB  0      0          127.0.0.1:fpo-fns       127.0.0.1:52880     users:(("server_game",pid=109301,fd=4))
\end{Verbatim}
Как и ожидалось, мы видим один слушающий сокет, и один установленный. При этом \texttt{127.0.0.1:fpo-fns} -- данные сервера, а \texttt{127.0.0.1:52880} -- данные клиента.

\section*{7. Множество подключений}
\addcontentsline{toc}{section}{7. Множество подключений}
\textbf{Задача:} Проделайте все заново, но запускайте не одно клиентское приложение (в виде \texttt{telnet}), а несколько экземпляров с разных терминалов, и попытайтесь работать с них одновременно.

Проанализируйте, как сервер будет обслуживать запросы в этом случае.

\textbf{Эксперимент:} После запуска двух дополнительных клиентов, мы видим, что они находится в состоянии блокировки на операции \texttt{connect}, и не получают сообщений от сервера.

\begin{Verbatim}[frame=single]
    smart@thinkpad$ ss -atp | grep game
    State  Recv-Q Send-Q Local Address:Port       Peer Address:PortProcess
    LISTEN 2      5            0.0.0.0:fpo-fns         0.0.0.0:*         users:(("server_game",pid=109301,fd=3))
    ESTAB  0      0          127.0.0.1:fpo-fns       127.0.0.1:52880     users:(("server_game",pid=109301,fd=4))
\end{Verbatim}
Состояние сокета показывает очередь на слушающем сокете длинной в 2.

\section*{8. Исправление игры}
\addcontentsline{toc}{section}{8. Исправление игры}
\textbf{Задача:} Модифицируйте программу \texttt{server\_game.cpp} так, чтобы запросы от каждого из клиентов могли обслуживаться конкурентно, путем запуска для каждого нового соединения собственного нового процесса на сервере или потока. Проанализируйте, как обслуживаются запросы в случае конкурентной схемы работы сервера.

Возможно также улучшить качество самой игровой функции \texttt{guess\_word()} сервера.

\textbf{Ход решения:}

Обработка подключений производится в отдельном процессе
\lstinputlisting[language=C++, caption={Фрагмент исходного кода модифицированного файла игры}, firstline=60, lastline=64]
{../Tasks/2_L4_Socket_sample/source_files/server_game_upd.cpp}

Исправлены условия для определения победы
\lstinputlisting[language=C++, caption={Фрагмент исходного кода модифицированного файла игры}, firstline=73, lastline=80]
{../Tasks/2_L4_Socket_sample/source_files/server_game_upd.cpp}

\textbf{Эксперимент:}

Состояние сокета показывает успешную работу с тремя клиентами.
\begin{Verbatim}[frame=single]
    smart@thinkpad$ ss -atp | grep game
    State  Recv-Q Send-Q Local Address:Port       Peer Address:PortProcess
    LISTEN 0      5            0.0.0.0:fpo-fns         0.0.0.0:*           users:(("server_game",pid=113319,fd=3),("server_game",pid=113317,fd=3),("server_game",pid=113310,fd=3),("server_game",pid=113306,fd=3))
    ESTAB  0      0          127.0.0.1:fpo-fns       127.0.0.1:52976       users:(("server_game",pid=113310,fd=4))                                                                                                
    ESTAB  0      0          127.0.0.1:fpo-fns       127.0.0.1:52978       users:(("server_game",pid=113317,fd=4))                                                                                                
    ESTAB  0      0          127.0.0.1:fpo-fns       127.0.0.1:56378       users:(("server_game",pid=113319,fd=4))
\end{Verbatim}

Победная ситуация успешно обрабатывается, имя хоста определяется правильно.
\begin{Verbatim}[frame=single]
    smart@thinkpad$ nc localhost 1066
    Playing on host: thinkpad:
    
     -----   12
    g
     g----   12
    r
     gr---   12
    e
     gree-   12
    n
     green   12
    You ween! Congrats!
\end{Verbatim}